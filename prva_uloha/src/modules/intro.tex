\section{Zadanie : 28}

\subsection{Názov projektu: Nemocnice}

\noindent\textbf{Zadanie:} \\
\indent Navrhněte IS malé nemocnice, který by měl poskytovat základní údaje o pacientech, lékařích, odděleních, vyšetřeních
pacientů, podávaných lécích apod. Zaměřte se i na otázku ochrany dat tak, aby měl každý lékař přístup pouze k potřebným
údajům.

\subsection{Riešenie - use case diagram}
Use case diagram alebo diagram použitia, vyjadruje ktorí aktéri a aké akcie sú potrebné v systéme.
Náš návrh pozostáva z aktérov:
\begin{outline}
    \1 Primár
    \2 Určuje zákroky a ich presuny, asistujúcich lekárov a sestry
    \1 Vedúci lekár
    \2 Vedie oddelenie, môže určiť zákrok avšak musí byť schválený jedine primárom
    \1 Ambulantný lekár
    \2 Lekár v ambulancii, príjma pacientov, vyšetruje ich a podľa toho určuje hospitalizáciu
    \1 Doktor
    \2 Viď popis nižšie
    \1 Zdravotná sestra
    \2 Pravá ruka lekára, asistuje lekárom, avšak z hľadiska systému nemá takmer žiadne právomoci
    \2 Vie zapísať pacienta do systému, zobraziť si lekárov na oddelení a zobraziť si údaje o liekoch
\end{outline}

Doktor reprezentuje "bežných"  lekárov v nemocnici.
Aktéri lekárov disponujú rovnakými právomocami a v rámci systému rovnakými akciami ako doktor.

A to sú
\begin{itemize}
    \item Zobraziť si pacienta
    \item Zapísať vyšetrenie
    \item Zobraziť vyšetrenie
    \item Zobraziť zákrok
    \item Predpísať lieky
    \item Zobraziť si informácie o liekoch
\end{itemize}

Aktéra všeobecného lekára sme v návrhu neuvažovali nakoľko sme mali navrhovať use\_case pre nemocnicu a všeobecný lekár
sídli vždy mimo nemocnicu.
\\\\
\textit{Návrh use\_case grafu}~\ref{fig:use_case}

\subsection{Riešenie - er diagram}
ER Diagram alebo aj entity relatonship diagram, vyjadruje dátový tok v rámci entít.
V našom prípade sme vychádzali z entít

\begin{itemize}
    \item Zdravotník
    \item Doktor
    \item Sestra
    \item Pacient
    \item Oddelenie
    \item Hospitalizácia
    \item Vyšetrenie
    \item Liek na predpis
    \item Zákrok
\end{itemize}

\noindent\textbf{Pričom:} \\
\indent Entita zdravotník reprezentuje akéhokoľvek zamestnanca v nemocnici. Pomocou atribútu pracovný status rozlišujeme medzi staničnými setrami,
všobecnými lekármi, primármi a pod. \\
Pomocou generalizácie ich ďalej delíme na sestry a doktorov. Pričom jediný rozdiel medzi sestrou a doktorom je atestácia, čož
je vyjadrené pomocou generalizácie a atribútu atestácia.

Zdravotník pracuje na jednom oddelení v rámci ktorého smie vykonávať vyšetrenia a zákroky, a účastniť sa na hospitalizácii.
Hoci hospitalizácia v preklade znamená "prijatie pacienta na lôžko", sú prípady kedy je prítomný doktor a či sestra. Čož
je vyjadrené vzťahom.

Entita Doktor má možnoť predpisovať lieky.

Entita pacient v našom prípade reprezentuje pacienta S zdravotnou knihou kde sa nachádzajú spisy o vštkých vyšetreniach a zákrokoch ktoré pacient
absolvoval.
Pacient sa hospitalizáciou začlenuje do systému pročom absolvuje vyšetrenia, zákroky a sú mu predpísané lieky

Entita Vyšetrenie obsahuje rovnaké údaje ako zákrok, čo je značené ďaľsou generalizáciou.
Vyšetrení sa zúčastní pacient, spolu so setrami a doktormi, resp. zdravotníkmi.

Entita Liek na predpis je striktne viazaná na pacienta, preto má vlastnosť  typu slabá.

Entita oddelenie rezprezentuje oddelenie a entita hospitalizácia reprezentuje príjem hospitalizovaného pacienta do nemocnice.
\\\\
\textit{Návrh er\_diagramu}~\ref{fig:er_diagram}
\newpage


